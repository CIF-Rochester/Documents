\documentclass[12pt]{amsart}

\title{Computer Interest Floor Constitution}

\begin{document}
\maketitle

\section {Membership}
All members have the following rights and responsibilities:
\begin{itemize}
	\item Attendance at Member Meetings (required for full members, encouraged for associates)
	\item Accumulation of Housing Points
	\item Notification of all public events
	\item Making Nominations
	\item Creating Petitions
\end{itemize}
	\subsection {Types of Membership}
		\subsubsection {Full Membership}
		Full Members have the following rights:
		\begin{itemize}
			\item Voting
			\item Increased Housing Points/Priority
			\item Access to Members-Only Events
		\end{itemize}
		Full Members must pay semesterly dues!
		
		\subsubsection {Associate Membership}
		Associate Membership is a form of official association with CIF, intended to provide a simple way for interested parties to stay up to date with CIF activities without needing to pay dues or participate as fully as Full Members.

Associate Members may be only be selected to live on floor if the floor has fewer Full Members than available spaces. If living on floor, an Associate must apply for Full status and be subject to those responsibilities.
	\subsection {Acceptance of Members}
	Anyone who wishes to become a member must apply through the membership application provided on the website. Applications are reviewed on a case-by-case basis. 
	
	To be accepted, applicants must be interviewed at an open board meeting. If at least four board members vote in favor of an applicant, they are granted full membership. Board members may decide to offer the applicant an associate membership, with the understanding that their application will be re-evaluated during the semester.

Off-floor members may change their membership level (from full to associate, associate to affiliate, or leave entirely), provided they notify Board. Board approval is required for a change in status of an on-floor member.
	\subsection{Affiliation}
	If somebody wishes to hear about floor events but does not care to participate in the running of floor, they can ask to be an affiliate and will only be added to the appropriate mailing list.
	\subsection {Floor Dues}
	At the first board meeting of the spring semester, the board must decide on the amount of dues for the next two semesters and the deadline for spring semester dues. At the first board meeting of the fall semester the board must decide on the deadline for paying fall semester dues.

	Floor members who cannot pay their dues may request a waiver or reduction, with explanation. The waiver or reduction must be approved by a majority of board. 
	
	Floor members may pay their dues in the form of improvements to floor, if and only if they clear it in advance with the Secretary-Treasurer and provide receipts for all money spent.

	A floor member who does not pay dues by the deadline set, unless they have been given a waiver or reduction, is reduced to associate membership for the remainder of the semester and therefore losing all voting rights and access to Member-Only Events.
\section {Elections and Officers}
Officers officially start their term at the beginning of the spring semester and end it at the end of the following fall semester.
	\subsection {Nominations}
	Nominations take place at the first floor meeting of November. All nominations must be seconded, and then accepted or declined by the nominee within a one-week period following the meeting. Nominations may continue electronically for this entire period, and must be seconded and accepted. Once the nomination period is over, the Secretary-Treasurer must notify all members with the list of candidates and offices.

	Requirements for nomination:
	\begin{itemize}
		\item The candidate must be a full member
		\item The candidate must be planning to attend U of R until the next round of elections
		\item The candidate must be planning to live on floor or in reasonable proximity to floor until the next round of elections
	\end{itemize}

	Candidates may campaign during the week between nominations and elections, but no floor funds may be used for this purpose.
	\subsection {Elections}
	Elections are held two weeks after the meeting at which nominations were made, preferably at a regularly scheduled floor meeting. 

	Positions are voted upon in order of precedence (the order in which they are listed below). Candidates may run for multiple positions, but may only hold one. Once elected, a candidate must drop out of any remaining races for which they are nominated.

	Before each position is voted upon, candidates will be permitted five minutes in which to address the members, and five minutes in which to field questions from the members.

	Officers are elected by simple majority of voting members present at the election. If no candidate receives a majority vote, a runoff election will be held between the two candidates with the most votes. Candidates may not vote for any position for which they are running. The current Chair does not vote except to break a tie between two candidates.

	The votes will be counted by both the current Secretary-Treasurer and an impartial representative of the University of Rochester Office of Residential Life.
	\subsection {Emergency Elections}
	If for some reason one or more officers are no longer able to fill their positions, the board may call for emergency elections within a compressed time-frame. They may shorten the nomination period and the time between nominations and elections to as little as three days each. Elections shall otherwise proceed as above.
	\subsection {Officer Duties}
	Each officer must have explicit Board approval to assume any power or responsibility for CIF affairs not laid out in this constitution.
		\subsubsection {Chair}
The Chair is responsible for:
\begin{itemize}
	\item Being the default contact between CIF and outside organizations
	\item Overseeing the Board to make sure that every officer is fulfilling their duties
	\item Acting as the main point of contact between CIF and Residential Life
	\item Ensuring that CIF is fulfilling the requirements set forth in the Expectations for Excellence
	\item Calling and presiding over meetings of both Board and members
	\item Forming committees not specifically under the jurisdiction of another officer
	\item Seeing that said committees accomplish their goals
	\item Ensuring that CIF is adequately represented on COG 
\end{itemize}
		\subsubsection {Logistics Director}
The Logistics Director is responsible for:
\begin{itemize}
	\item Handling membership issues in conjunction with the Chair
	\item Planning and executing social events, both internal and open to the Towers or University communities
	\item Coordinating with the Tech Services Director for tech seminars and similar events
	\item Coordinating the application process
	\item Accepting and responding to applications for membership
	\item Coordinating the housing process in the spring semester, in conjunction with the rest of board
	\item Updating and maintaining membership files (information including name, date accepted, contact information, participation points, and housing eligibility) in conjunction with the Secretary-Treasurer
	\item Surveying all members at the beginning of the spring semester to see who is planning to live on CIF in the next year; as a result of this, setting a recruitment goal to bring floor to capacity
	\item Coordinating CIF events and RA programs to make sure they do not conflict
	\item Helping groups of members organize and execute events
	\item Aiding the Tech Maintenance Director or Webmaster in updating and maintaining the mailing lists
\end{itemize}
		\subsubsection {Tech Services Director}
The Tech Services Director is responsible for:
\begin{itemize}
	\item Overseeing Help@CIF volunteers and responding to clients
	\item Overseeing and maintaining the lab as a study space
	\item Maintaining the printers in the lab
	\item Maintaining the website front-end in the absence of a Webmaster
	\item Organizing tech seminars and similar events, in coordination with the Logistics Director
\end{itemize}
		\subsubsection {Tech Maintenance Director}
The Tech Maintenance Director is responsible for:
\begin{itemize}
	\item Maintaining and updating the servers and website back-end
	\item Maintaining internet access
	\item Updating and maintaining the mailing lists in the absence of a Webmaster, with the aid of the Logistics Director
	\item Maintaining the lab card reader
	\item Maintaining and updating the lab computers
\end{itemize}
		\subsubsection {Secretary-Treasurer}
The Secretary-Treasurer is responsible for:
\begin{itemize}
	\item Recording minutes of all meetings and sending them to the mailing lists
	\item Notifying Members and Affiliates of upcoming events and other announcements
	\item Keeping CIF records in a secure location (records for the current semester may be kept in the Secretary-Treasurer's room), to be made available to anybody who requests to see them
	\item Updating the website with events, announcements, and minutes, in the absence of a Webmaster
	\item Collecting membership dues
	\item Monitoring distribution of CIF funds
	\item Keeping track of attendance at all CIF events and meetings
	\item Updating and maintaining membership files (information including name, date accepted, contact information, participation points, and housing eligibility) in conjunction with the
	\item Creating a budget based on the prior semester's expenditures, which should be presented to Board at the second Board meeting of the semester.
	\item Creating a summary, at the end of each semester, of expenditures over the course of the semester, to be used in making the next budget
\end{itemize}
		\subsubsection{Notes:}
Each officer has the power to appoint and oversee a committee to help them with any aspect of their job.

The Secretary-Treasurer may authorize expenditures of up to \$25 at their discretion, provided at least two other board members are notified within one week or before the next such expenditure; if such notification is not provided, a majority vote of the remainder of Board can revoke this spending power.

At least three members of Board must approve expenditures up to \$100.

Expenditures of over \$100 must be approved by a majority vote of full members.

A unanimous Board has the power to tell the Tech Maintenance Director or other sysadmin(s) to lock a member's account(s) if that member is suspected of computer-related misdeeds. If an officer is the person suspected of such misdeeds, they are not considered a member of Board for any discussion or votes on the matter.
\section {Appointed Positions}
{\Large Board cannot force people to do these things. The descriptions should be written down and posted publically}
Board may appoint members to do certain tasks. Appointed officers must be voted for by at least four members of board, and should have well-defined duties. Board is highly encouraged to appoint at least the following two positions:
	\subsection{Member Liaison}
	The Member Liaison exists to bridge the gap between board and the rest of the membership. The Member Liaison is essentially a non-voting member of the board: they sit in on board meetings so they can convey to other members what board is working on and talking about, and so that they can bring up the concerns of other members to board in an organized fashion. This position cannot be held by a member of board.
	\subsection{Webmaster}
	The Webmaster is expected to update and maintain the CIF website. They should post minutes of Member and Board meetings, announcements about newly accepted members, and announcements of events, whether social or technical.
	
	The Webmaster should also update and maintain the mailing lists, under the direction of the Tech Maintenance Director and the Logistics Director.
	\subsection{External Representatives}
	To ensure consistency of representation, the Chair shall be the default representative to any external party. If necessary, the Board shall appoint floor members as additional or alternate representatives. Such appointments must be approved by a majority of Board. Representatives must report to Board after any meeting in which they represent CIF, and provide minutes if possible. Non-Board representatives cannot enter into agreements with or make commitments on CIF's behalf without prior Board approval.
 
\section{Meetings}
\subsection {Member Meetings}
Member meetings should be held at least once every two weeks, at a default time to be set by board at the beginning of the semester. 

Member meetings may be called in three ways:
\begin{enumerate}
	\item The Logistics Director may call a meeting for the default time with at least 48 hours' advance notice
	\item One-fourth of the voting members may sign a petition calling for a meeting to be held at any time with at least 48 hours' advance notice
	\item Any board member may call an emergency meeting for any time with any amount of advance notice - but meetings called in this way cannot be made mandatory with less than 48 hours' advance notice
\end{enumerate}

The Chair presides over member meetings, unless they are unable, in which case the next officer in order of succession presides. Each officer is given an opportunity to report on current developments and plans within their domain.

A quorum for member meetings is one-half of the voting members. A quorum is required for any vote, and a proposal is approved by a member vote if a majority of those present vote in favor of it.
If a member is not present for three consecutive meetings without prior notification and permission of board, they will lose voting privileges until they have attended three consecutive meetings.
\subsection {Board Meetings}
Board meetings should be held every week, at a default time to be set by the board at the beginning of the semester.

Board meetings may be called in three ways:
\begin{enumerate}
	\item The Chair may call a board meeting for the default time with at least 48 hours' advance notice
	\item Any two board members may call a board meeting for the default time with at least 48 hours' advance notice
	\item All members of board may agree to meet at any time
\end{enumerate}

The Logistics Officer will notify the members of the times and locations of planned board meetings. A quorum for board meetings is four of the members, and is required for any board vote.

If a board meeting is called and a quorum cannot be met, the meeting must be rescheduled

\section{Participation Points}
Every member is expected to participate in CIF events, to help foster community, and to help promote the floor to the University community as a whole. At the beginning of every semester, Board chooses a quota for participation points, which are awarded as follows:
\begin{itemize}
	\item 1 point for attending an event
	\item 2 points for helping with publicity for an event
	\item 3 points for helping the Logistics Director produce and run an event
	\item A number of points to be decided by Board for appointed and representative positions
	\item A number of points to be decided by committee heads for actively participating in a committee, subject to Board approval
\end{itemize}
Points are awarded only for events announced with at least 48 hours notice.  Point accumulation must be reported to a member of Board or recorded on a sign-in sheet.

If a Member doesn't meet the point quota for a semester, they can request a waiver from Board; if it is not granted, they will be reduced to the next membership level (Full to Associate, Associate to Affiliate).

Board members are exempt from this requirement unless they are impeached or resign.

Members on leave (such as those studying abroad) will be exempt from participation point quotas.

The Secretary-Treasurer should keep running point totals in membership records and make them available to members upon request. Participation point totals reset at the beginning of every semester.

\section {Overriding Board and Board Member Decisions}
\subsection{Overriding Board Members}
If any officer wishes to dispute a decision or action made by another officer, they may take either of two actions, and must specifically describe the issue and their position on it.
\begin{enumerate}
	\item Call a board meeting in conjunction with another officer to discuss the issue and call for a vote to override the decision or action in question by majority vote
	\item Have a majority of officers sign a petition to override the decision or action in question
\end{enumerate}
If a majority of Board does not agree to sign such a petition or attend such a meeting, the decision or action cannot be overridden.

If a decision or action is successfully overridden, the officer in question may take one of two actions:
\begin{enumerate}
	\item Comply with the decision of Board within one day or as soon as possible
	\item Petition to call a members meeting at which to call for a vote; if the officer does not collect the signatures of one quarter of the members within one day, they must comply with the decision of the board.
\end{enumerate}
If such a meeting is called and a quorum is not present, three-quarters of those present may elect to postpone the meeting, only once, for at most a week. If there is no successful postponement, the Board's decision is upheld and the officer must comply with it.

If such a meeting is called and a quorum is present, a majority vote of those present is required to overrule the decision to override the officer. If the vote succeeds, the Board's override is overridden and the original decision is upheld.
\subsection{Overriding Board}
Any member who disagrees with a decision made by the board can call for a member vote at a member meeting. If a quorum is present, then a majority vote will override the board's decision. Any board decision can be overridden in this manner.
	\subsection {Resignation}
In the event that a Board member or appointed officer becomes, in their own judgment, unable to fulfill their duties, they may resign from their position. If the resigned is an elected officer, their resignation must be immediately followed by nominations for an emergency election to fill the position, unless a different officer-elect already exists for the position. If the resigned is an appointed officer, Board is encouraged to appoint a replacement as quickly as possible.
	\subsection {Impeachment}
A vote of impeachment can be brought by any member by means of a petition specifying the reason(s) for impeaching the board member in question; such a petition must gain the signatures of at least half of the full members. It must also specify the time and date of the floor meeting at which the impeachment vote will be held.

At the floor meeting at which the impeachment vote is to be held, all board members must be allowed to speak, as well as the member who is bringing the vote, and a limited number of other members. The board member who is the subject of the vote may not preside; if necessary, the next officer in the order of succession presides. 

A vote of impeachment is successful if a simple majority of members (not only a simple majority of those present) vote to impeach the board member in question. A successful impeachment vote brought against an elected officer must be followed immediately by nominations for an emergency election to be held at a time determined by the remaining Board, unless a different officer-elect already exists for the position. If the impeached is an appointed officer, Board is encouraged to appoint a replacement as quickly as possible.
	\subsection {Resolving Resignation or Impeachment}
The order of succession is as follows:
\begin{enumerate}
	\item Chair
	\item Logistics Director
	\item Tech Services Director
	\item Tech Maintenance Director
	\item Secretary-Treasurer
\end{enumerate}

If the Chair resigns or is impeached, the next officer in the order of succession will take on their responsibilities as acting Chair until an emergency election is held, continuing down the chain if this officer also resigns or is impeached.

If any officer other than Chair resigns or is impeached, the Chair is responsible for doing or delegating the duties of that board member until an emergency election can be held.

If the Secretary-Treasurer is impeached, the votes in the emergency election are counted by the Chair or acting Chair.

If an officer resigns or is impeached within the last week of classes during the spring semester, an election to fill the vacancy is not held until the fall, following the rules for an emergency election.

If a current board member is elected to fill a vacancy left by impeachment or resignation, they must vacate their current position to fill the new one, in which case another election must be held. Nominations for their previous position may be made immediately after the current board member is nominated to fill the vacancy, and a new officer elected at the same emergency election meeting.
\section {Housing}
	\subsection {Housing Points}
	Members are awarded housing points as follows:
\begin{itemize}
	\item 4 points for each semester they have been a dues-paying member
	\item 1 point for each semester they have been an associate member
	\item .75 points for each semester they have been active at the University, with a maximum of eight semesters, including time spent abroad
	\item .5 points per semester spent active at another institution, not to exceed eight when combined with semesters at this university
	\item .5 points for the winner of the Nauticock Award
\end{itemize}

Full members who are studying abroad will receive the associate member point total for each of their semester(s) abroad. If a member voluntarily pays dues despite being abroad, they will receive 2 points.

Each semester in which a member has paid the full amount of dues is to count towards their point total; if someone is accepted as an associate member more than halfway through a semester, that semester does not count towards their housing point total.

Housing points must be tabulated by the Logistics Director at least two weeks before the housing due date. The numbers must then be verified, recorded, and made available to all members by the Secretary-Treasurer within 48 hours. If any member disputes their posted points, they should address it to Board within 48 hours.
		\subsubsection {Nauticock Award}
This award is given to recognize a CIF member who has gone above and beyond the expected levels of participation and who has dedicated extraordinary amounts of time and resources to the improvement of CIF as a whole. Once per spring, members will nominate possible recipients and then vote on them as per officer election procedures. The winner receives the Nauticock Award.
	\subsection {Room Assignments}
Members with the most points receive the highest priority when rooms are assigned. If a group of people wish to live together, their points are totaled to determine their priority in room assignment.

All groups of people wishing to live together must fit into an available logical room grouping, following requirements imposed by Residential Life.

In the event of a tie, the following tie-breaking points are calculated:
\begin{itemize}
	\item 3 points per semester spent with a roommate in a double
	\item 2 points per semester spent as a non-resident full member
\end{itemize}
If there is still a tie, it shall be decided by random lot.
	\subsection {Selection of New Resident Members}
If a new member wishes to move into an available room mid-year, they must receive approval of both the Board and the residents of the room grouping in which they will be living to be allowed to move in.
\section {Conduct, Discipline, and Status Change of Floor Members}
	\subsection {Probation}
Any member may bring a complaint to the Board and request that another member be placed on probation. The request must be reviewed at an open board meeting whose time and purpose are announced to the floor with at least 48 hours' advance notice. Any member may attend this meeting and speak on the topic of placing the member on probation. The floor member being considered may attend the meeting but not the vote; Board may decide to make the vote confidential.

A majority vote of Board is required to place a member on probation, at which time they must provide a timeline and requirements that must be met for the member to be let off probation. Additional meetings may be held to check on the member or re-evaluate the terms of probation.

Members who are on probation are expected to attend all meetings and activities, or else give a good reason for their absence. Members who are on probation but fail to comply with the terms will face expulsion.
	\subsection {Expulsion}
A member who violates probation is subject to expulsion. If the board determines that a member has violated probation, they must be given written, non-electronic notice that they have done so and that they must appear at the next board meeting to explain their actions. If they do not appear at the board meeting, having been given at least 48 hours' advance notice of its time and location, and cannot offer any explanation for their absence, they are expelled from CIF.

If the member offers an explanation of their actions to Board, then Board must hold a vote to decide if the explanation is acceptable. If four Board members vote that it is not, the member in question is expelled from CIF.

If a member remains on probation for an entire semester, they are brought to a hearing at a member meeting where they offer an explanation of their actions. The members will vote on the member's expulsion. If two-thirds of total membership or more vote in favor of expulsion, the member is expelled from CIF.

A member who has been expelled must be notified of their expulsion in person by the Chair. If the expulsion was voted on by Board, the Chair must ask the member if they wish to appeal the decision to the members. If they do not wish to appeal, their membership is immediately terminated. If they do wish to appeal, the appeal must be made at the next member meeting following all rules for overriding Board decisions. If the decision is not overridden, their membership is terminated at the end of the meeting.

If a resident member is expelled, they are expected to move off of floor in a timely manner. They must contact Residential Life about applying for a room change, with the assistance of the Chair, who is responsible for communicating and coordinating with the Residential Life Area Coordinator.

If a member is expelled before the halfway point of the semester, their dues for that semester will be refunded.
	\subsection {Withdrawal}
If a member decides that they no longer wish to be a member, they may withdraw from membership status. If a resident member decides to withdraw, they should arrange to move off floor in a timely manner.

If a member withdraws before the halfway point of the semester, their dues for that semester will be refunded.
\section {Procedures for Constitutional Amendment and Ratification}
	\subsection {Temporary Amendment}
During a member meeting, a member may propose a temporary amendment, explaining the special circumstances necessitating the amendment and the duration of the amendment's effect, which may not be longer than one year. A member vote will take place immediately; the amendment will pass with a two-thirds majority vote of total membership.
	\subsection {Amendment}
An amendment to the constitution will modify the actual text of the document, rather than being appended.

Any full member may draft and propose an amendment to the constitution. An amendment must be drafted before it is officially proposed. During a floor meeting, the member may propose the amendment, explain why they think it is necessary, and call for a vote on that amendment at the next floor meeting or in no less than two weeks' time	. They must make the amended text of the constitution available as soon as possible following the proposal.

An amendment to the constitution passes if a two-thirds majority of total dues-paying membership votes in favor.
	\subsection {Ratification}
This constitution may be replaced by a new constitution or a new draft of this constitution. The draft of the new or revised constitution, as well as a copy of this constitution, must be presented to:
\begin{itemize}
	\item All full members of CIF
	\item The Area Coordinator for Residential Life
\end{itemize}
The listed parties have two weeks after the presentation of the draft to make any suggestions they desire and to debate the merits and deficits of the draft.

After two weeks a vote will be held at a member meeting. If three-quarters of the total CIF membership vote in favor of the new constitution, it is considered ratified and goes into effect at a time specified by the Board.

\end{document}