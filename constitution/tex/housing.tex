\section{Housing}
\label{sec:housing}



	\subsection{Housing Points}

	Members are awarded housing points as follows:
	\begin{itemize}
		\item 4 points for each semester they have been a dues-paying member
		\item 1 point for each semester they have been an associate member
		\item 0.75 points for each semester they have been active at the University, with a maximum of eight semesters, including time spent abroad
		\item 0.5 points per semester spent active at another institution, not to exceed eight when combined with semesters at this university
		\item 0.5 points for the winner of the Nauticock Award
	\end{itemize}

	Full members who are studying abroad will receive the associate member point total for each of their semester\,(s) abroad. If a member voluntarily pays dues despite being abroad, they will receive 2 points.

	Each semester in which a member has paid the full amount of dues is to count towards their point total; if someone is accepted as an associate member more than halfway through a semester, that semester does not count towards their housing point total.

	Housing points must be tabulated by the Logistics Director at least two weeks before the housing due date. The numbers must then be verified, recorded, and made available to all members by the Secretary-Treasurer within 48 hours. If any member disputes their posted points, they should address it to Board within 48 hours.



		\subsubsection{Nauticock Award}

		This award is given to recognize a CIF Member who has gone above and beyond the expected levels of participation and who has dedicated extraordinary amounts of time and resources to the improvement of CIF as a whole. Once per spring, Members will nominate possible recipients and then vote on them as per officer election procedures. The winner receives the Nauticock Award. A person can only win the Nauticock Award once.



	\subsection{Room Assignments}

Members with the most points receive the highest priority when rooms are assigned. If a group of people wish to live together, their points are totaled to determine their priority in room assignment.

All groups of people wishing to live together must fit into an available logical room grouping, following requirements imposed by Residential Life.

In the event of a tie, the following tie-breaking points are calculated:
\begin{itemize}
	\item 3 points per semester spent with a roommate in a double
	\item 2 points per semester spent as a non-resident full member
\end{itemize}
If there is still a tie, it shall be decided by random lot.



	\subsection{Selection of New Resident Members}

If a new member wishes to move into an available room mid-year, they must receive approval of both the Board and the residents of the room grouping in which they will be living to be allowed to move in.
