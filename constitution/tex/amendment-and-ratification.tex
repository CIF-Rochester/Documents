\section {Procedures for Constitutional Amendment and Ratification}
\label{sec:amendment-and-ratification}



	\subsection {Temporary Amendment}

During a member meeting, a member may propose a temporary amendment, explaining the special circumstances necessitating the amendment and the duration of the amendment's effect, which may not be longer than one year. A member vote will take place immediately; the amendment will pass with a two-thirds majority vote of total membership.



	\subsection {Amendment}

An amendment to the constitution will modify the actual text of the document, rather than being appended.

Any full member may draft and propose an amendment to the constitution. An amendment must be drafted before it is officially proposed. During a floor meeting, the member may propose the amendment, explain why they think it is necessary, and call for a vote on that amendment at the next floor meeting or in no less than two weeks' time	. They must make the amended text of the constitution available as soon as possible following the proposal.

An amendment to the constitution passes if a two-thirds majority of total dues-paying membership votes in favor.



	\subsection {Ratification}

This constitution may be replaced by a new constitution or a new draft of this constitution. The draft of the new or revised constitution, as well as a copy of this constitution, must be presented to:
\begin{itemize}
	\item All full members of CIF
	\item The Area Coordinator for Residential Life
\end{itemize}
The listed parties have two weeks after the presentation of the draft to make any suggestions they desire and to debate the merits and deficits of the draft.

After two weeks a vote will be held at a member meeting. If three-quarters of the total CIF membership vote in favor of the new constitution, it is considered ratified and goes into effect at a time specified by the Board.
