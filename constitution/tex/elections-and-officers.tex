\section{Elections and Officers}
\label{sec:elections-and-officers}

Officers officially start their term at the beginning of the spring semester and end it at the end of the following fall semester.



	\subsection{Nominations}

	Nominations take place at the first floor meeting of November. All nominations must be seconded, and then accepted or declined by the nominee within a one-week period following the meeting. Nominations may continue electronically for this entire period, and must be seconded and accepted. Once the nomination period is over, the Secretary-Treasurer must notify all members with the list of candidates and offices.

	Requirements for nomination:
	\begin{itemize}
		\item The candidate must be a full member
		\item The candidate must be planning to attend U of R until the next round of elections
		\item The candidate must be planning to live on floor or in reasonable proximity to floor until the next round of elections
	\end{itemize}

	Candidates may campaign during the week between nominations and elections, but no floor funds may be used for this purpose.



	\subsection {Elections}

	Elections are held two weeks after the meeting at which nominations were made, preferably at a regularly scheduled floor meeting. 

	Positions are voted upon in order of precedence (the order in which they are listed below). Candidates may run for multiple positions, but may only hold one. Once elected, a candidate must drop out of any remaining races for which they are nominated.

	Before each position is voted upon, candidates will be permitted five minutes in which to address the members, and five minutes in which to field questions from the members.

	Officers are elected by simple majority of voting members present at the election. If no candidate receives a majority vote, a runoff election will be held between the two candidates with the most votes. Candidates may not vote for any position for which they are running. The current Chair does not vote except to break a tie between two candidates.

	The votes will be counted by both the current Secretary-Treasurer and an impartial representative of the University of Rochester Office of Residential Life.



	\subsection {Emergency Elections}

	If for some reason one or more officers are no longer able to fill their positions, the board may call for emergency elections within a compressed time-frame. They may shorten the nomination period and the time between nominations and elections to as little as three days each. Elections shall otherwise proceed as above.



	\subsection {Officer Duties}

	Each officer must have explicit Board approval to assume any power or responsibility for CIF affairs not laid out in this constitution.



		\subsubsection {Chair}

The Chair is responsible for:
\begin{itemize}
	\item Being the default contact between CIF and outside organizations
	\item Overseeing the Board to make sure that every officer is fulfilling their duties
	\item Acting as the main point of contact between CIF and Residential Life
	\item Ensuring that CIF is fulfilling the requirements set forth in the Expectations for Excellence
	\item Calling and presiding over meetings of both Board and members
	\item Forming committees not specifically under the jurisdiction of another officer
	\item Seeing that said committees accomplish their goals
	\item Ensuring that CIF is adequately represented on COG 
\end{itemize}



		\subsubsection {Logistics Director}

The Logistics Director is responsible for:
\begin{itemize}
	\item Handling membership issues in conjunction with the Chair
	\item Planning and executing social events, both internal and open to the Towers or University communities
	\item Coordinating with the Tech Services Director for tech seminars and similar events
	\item Coordinating the application process
	\item Accepting and responding to applications for membership
	\item Coordinating the housing process in the spring semester, in conjunction with the rest of board
	\item Updating and maintaining membership files (information including name, date accepted, contact information, participation points, and housing eligibility) in conjunction with the Secretary-Treasurer
	\item Surveying all members at the beginning of the spring semester to see who is planning to live on CIF in the next year; as a result of this, setting a recruitment goal to bring floor to capacity
	\item Coordinating CIF events and RA programs to make sure they do not conflict
	\item Helping groups of members organize and execute events
	\item Aiding the Tech Maintenance Director or Webmaster in updating and maintaining the mailing lists
\end{itemize}



		\subsubsection {Tech Services Director}

The Tech Services Director is responsible for:
\begin{itemize}
	\item Overseeing Help@CIF volunteers and responding to clients
	\item Overseeing and maintaining the lab as a study space
	\item Maintaining the printers in the lab
	\item Maintaining the website front-end in the absence of a Webmaster
	\item Organizing tech seminars and similar events, in coordination with the Logistics Director
\end{itemize}

		\subsubsection {Tech Maintenance Director}



The Tech Maintenance Director is responsible for:
\begin{itemize}
	\item Maintaining and updating the servers and website back-end
	\item Maintaining internet access
	\item Updating and maintaining the mailing lists in the absence of a Webmaster, with the aid of the Logistics Director
	\item Maintaining the lab card reader
	\item Maintaining and updating the lab computers
\end{itemize}



		\subsubsection {Secretary-Treasurer}

The Secretary-Treasurer is responsible for:
\begin{itemize}
	\item Recording minutes of all meetings and sending them to the mailing lists
	\item Notifying Members and Affiliates of upcoming events and other announcements
	\item Keeping CIF records in a secure location (records for the current semester may be kept in the Secretary-Treasurer's room), to be made available to anybody who requests to see them
	\item Updating the website with events, announcements, and minutes, in the absence of a Webmaster
	\item Collecting membership dues
	\item Monitoring distribution of CIF funds
	\item Keeping track of attendance at all CIF events and meetings
	\item Updating and maintaining membership files (information including name, date accepted, contact information, participation points, and housing eligibility) in conjunction with the
	\item Creating a budget based on the prior semester's expenditures, which should be presented to Board at the second Board meeting of the semester.
	\item Creating a summary, at the end of each semester, of expenditures over the course of the semester, to be used in making the next budget
\end{itemize}



		\subsubsection{Notes:}

Each officer has the power to appoint and oversee a committee to help them with any aspect of their job.

The Secretary-Treasurer may authorize expenditures of up to \$25 at their discretion, provided at least two other board members are notified within one week or before the next such expenditure; if such notification is not provided, a majority vote of the remainder of Board can revoke this spending power.

At least three members of Board must approve expenditures up to \$100.

Expenditures of over \$100 must be approved by a majority vote of full members.

A unanimous Board has the power to tell the Tech Maintenance Director or other sysadmin(s) to lock a member's account(s) if that member is suspected of computer-related misdeeds. If an officer is the person suspected of such misdeeds, they are not considered a member of Board for any discussion or votes on the matter.
