\section{Elections and Officers}
\label{sec:elections-and-officers}

Officers officially start their term at the beginning of the spring semester and end it at the end of the following fall semester.



	\subsection{Nominations}

	Nominations take place at a floor meeting in late October. All nominations must be seconded, and then accepted or declined by the nominee within a one-week period following the meeting. Nominations may continue electronically for this entire period, and must be seconded and accepted. Once the nomination period is over, the Secretary-Treasurer must notify all Members with the list of candidates and offices.

	Requirements for nomination:
	\begin{itemize}
		\item The candidate must be a full member
		\item The candidate must be planning to attend the University of Rochester until the next round of elections
		\item The candidate must be able to fulfill all requirements of the position, including but not limited to attendance of board meetings, attendance of floor meetings, and all duties listed below
	\end{itemize}

	Candidates may campaign during the week between nominations and elections, but no floor funds may be used for this purpose.



	\subsection{Elections}

	Elections are held two weeks after the meeting at which nominations were made, preferably at a regularly scheduled floor meeting.

	Positions are voted upon in order of precedence (the order in which they are listed below). Candidates may run for multiple positions, but may only hold one. Once elected, a candidate must drop out of any remaining races for which they are nominated.

	Before each position is voted upon, candidates will be permitted five minutes in which to address the members, and five minutes in which to field questions from the members.

	Officers are elected by simple majority of voting members present at the election. If no candidate receives a majority vote, a runoff election will be held between the two candidates with the most votes. Candidates may not vote for any position for which they are running. The current Chair does not vote except to break a tie between two candidates.

	The votes will be counted by both the current Secretary-Treasurer and an impartial representative of the University of Rochester Office of Residential Life. In the event that the current Secretary-Treasurer is running for re-election, then the current Chair will count votes in their place.

	\subsection{Emergency Elections}

	If one or more officers are no longer able to fill their positions, the Board may call for emergency elections within a compressed time-frame. They may shorten the nomination period and the time between nominations and elections to as little as three days each. Elections shall otherwise proceed as above. In the interim, the responsibilities of that position are dictated as seen in Section~\ref{sub:resolving-resignation-or-impeachment}.



	\subsection{Officer Duties}

	Each officer must have explicit Board approval to assume any power or responsibility for CIF affairs not laid out in this constitution.



		\subsubsection{Chair}\label{ssub:chair}

The Chair is responsible for:
\begin{itemize}
	\item Being the default contact between CIF and outside organizations
	\item Overseeing The Board to make sure that every officer is fulfilling their duties
	\item Acting as the main point of contact between CIF and Residential Life
	\item Ensuring that CIF is fulfilling the requirements set forth in the Expectations for Excellence
	\item Calling and presiding over meetings of both The Board and Members
	\item Forming committees not specifically under the jurisdiction of another officer
	\item Seeing that said committees accomplish their goals
	\item Ensuring that CIF is adequately represented on the Special Interest Housing President’s Council (SIHPC)
	\item Updating the Chair Binder
\end{itemize}



		\subsubsection{Logistics Director}\label{ssub:logistics-director}

The Logistics Director is responsible for:
\begin{itemize}
	\item Handling membership issues in conjunction with the Chair
	\item Planning and executing social events, both internal and open to the Towers or University communities
	\item Coordinating with the Tech Services Director for tech seminars and similar events
	\item Email applicants and arrange an interview time with The Board
	\item Calculate member housing points for the Housing Meeting in the Spring
	\item Updating and maintaining Membership files (information including name, month of acceptance, NetID and/or alternate email address)
	\item Surveying all members at the beginning of the spring semester to see who is planning to live on CIF in the next year; as a result of this, setting a recruitment goal to bring floor to capacity
	\item Coordinating CIF events and RA programs to make sure they do not conflict
	\item Helping groups of members organize and execute events
	\item Aiding the Tech Maintenance Director in updating and maintaining the mailing lists
	\item Updating CIF's online calendar (via Google)
\end{itemize}



		\subsubsection{Tech Services Director}\label{ssub:tech-services-director}

Both Tech Directors are expected to assist one another in their duties. Tech Services Director is explicitly responsible for the following:
\begin{itemize}
	\item Overseeing Help@CIF volunteers and responding to clients
	\item Overseeing and maintaining the lab as a study space
	\item Maintaining the printer hardware in the lab
	\item Maintaining the content of the website
	\item Organizing dates and topics for tech seminars and other tech events
	\item Interfacing with Resnet for issues regarding the network or servers as they relate to the University
\end{itemize}



		\subsubsection{Tech Maintenance Director}\label{ssub:tech-maintenance-director}

Both Tech Directors are expected to assist one another in their duties. The Tech Maintenance Director is explicitly responsible for:
\begin{itemize}
	\item The Tech Maintenance Director is responsible for:
 
	\item Maintaining and updating the servers and website back-end
	\item Maintaining the frontend code and backend of the website
	\item Ensuring someone is available to fulfill each Help@CIF request
	\item Maintaining lab machine software deployments from servers
\end{itemize}



		\subsubsection{Secretary-Treasurer}\label{ssub:secretary-treasurer}

The Secretary-Treasurer is responsible for:
\begin{itemize}
	\item Keeping and protecting the CIF treasury
	\item Recording minutes of all meetings and sending them to the mailing lists
	\item Notifying Members and Affiliates of upcoming events and other announcements
	\item Keeping CIF records in a secure location (records for the current semester may be kept in the Secretary-Treasurer's room), to be made available to anybody who requests to see them
	\item Updating the website with announcements and minutes
	\item Collecting membership dues
	\item Monitoring distribution of CIF funds
	\item Keeping track of attendance at all CIF events and meetings
	\item Updating and maintaining membership files in relation to participation points and dues
	\item Creating a budget based on the prior semester's expenditures, which should be presented to Board at the second Board meeting of the semester.
	\item Creating a summary, at the end of each semester, of expenditures over the course of the semester, to be used in making the next budget
	\item Distributing housing point information within 48 hours of it being tabulated by the Logistics Director
	\item Keeping track of all titles, rights, and responsibilities of all current appointed positions
\end{itemize}



		\subsubsection{Notes:}

Each officer has the power to appoint and oversee a committee to help them with any aspect of their job.

The Secretary-Treasurer may authorize expenditures of up to \$25 at their discretion, provided at least two other board members are notified within one week or before the next such expenditure; if such notification is not provided, a majority vote of the remainder of Board can revoke this spending power.

At least three members of Board must approve expenditures up to \$100.

A unanimous agreement of The Board has the power to tell the Tech Maintenance Director or other sysadmin\,(s) to lock a Member’s account if that Member is suspected of computer-related misdeeds. If an officer is the person suspected of such misdeeds, they are not considered a Member of The Board for any discussion or votes on the matter.
