\section {Overriding Board and Board Member Decisions}
\label{sec:overriding-board-decisions}



\subsection{Overriding Board Members}

If any officer wishes to dispute a decision or action made by another officer, they may take either of two actions, and must specifically describe the issue and their position on it.
\begin{enumerate}
	\item Call a board meeting in conjunction with another officer to discuss the issue and call for a vote to override the decision or action in question by majority vote
	\item Have a majority of officers sign a petition to override the decision or action in question
\end{enumerate}
If a majority of Board does not agree to sign such a petition or attend such a meeting, the decision or action cannot be overridden.

If a decision or action is successfully overridden, the officer in question may take one of two actions:
\begin{enumerate}
	\item Comply with the decision of Board within one day or as soon as possible
	\item Petition to call a members meeting at which to call for a vote; if the officer does not collect the signatures of one quarter of the members within one day, they must comply with the decision of the board.
\end{enumerate}
If such a meeting is called and a quorum is not present, three-quarters of those present may elect to postpone the meeting, only once, for at most a week. If there is no successful postponement, the Board's decision is upheld and the officer must comply with it.

If such a meeting is called and a quorum is present, a majority vote of those present is required to overrule the decision to override the officer. If the vote succeeds, the Board's override is overridden and the original decision is upheld.



\subsection{Overriding Board}

Any member who disagrees with a decision made by the board can call for a member vote at a member meeting. If a quorum is present, then a majority vote will override the board's decision. Any board decision can be overridden in this manner.



	\subsection {Resignation}

In the event that a Board member or appointed officer becomes, in their own judgment, unable to fulfill their duties, they may resign from their position. If the resigned is an elected officer, their resignation must be immediately followed by nominations for an emergency election to fill the position, unless a different officer-elect already exists for the position. If the resigned is an appointed officer, Board is encouraged to appoint a replacement as quickly as possible.



	\subsection {Impeachment}

A vote of impeachment can be brought by any member by means of a petition specifying the reason(s) for impeaching the board member in question; such a petition must gain the signatures of at least half of the full members. It must also specify the time and date of the floor meeting at which the impeachment vote will be held.

At the floor meeting at which the impeachment vote is to be held, all board members must be allowed to speak, as well as the member who is bringing the vote, and a limited number of other members. The board member who is the subject of the vote may not preside; if necessary, the next officer in the order of succession presides. 

A vote of impeachment is successful if a simple majority of members (not only a simple majority of those present) vote to impeach the board member in question. A successful impeachment vote brought against an elected officer must be followed immediately by nominations for an emergency election to be held at a time determined by the remaining Board, unless a different officer-elect already exists for the position. If the impeached is an appointed officer, Board is encouraged to appoint a replacement as quickly as possible.



	\subsection {Resolving Resignation or Impeachment}

The order of succession is as follows:
\begin{enumerate}
	\item Chair
	\item Logistics Director
	\item Tech Services Director
	\item Tech Maintenance Director
	\item Secretary-Treasurer
\end{enumerate}

If the Chair resigns or is impeached, the next officer in the order of succession will take on their responsibilities as acting Chair until an emergency election is held, continuing down the chain if this officer also resigns or is impeached.

If any officer other than Chair resigns or is impeached, the Chair is responsible for doing or delegating the duties of that board member until an emergency election can be held.

If the Secretary-Treasurer is impeached, the votes in the emergency election are counted by the Chair or acting Chair.

If an officer resigns or is impeached within the last week of classes during the spring semester, an election to fill the vacancy is not held until the fall, following the rules for an emergency election.

If a current board member is elected to fill a vacancy left by impeachment or resignation, they must vacate their current position to fill the new one, in which case another election must be held. Nominations for their previous position may be made immediately after the current board member is nominated to fill the vacancy, and a new officer elected at the same emergency election meeting.
