\section {Conduct, Discipline, and Status Change of Floor Members}
\label{sec:conduct-and-discipline}



	\subsection {Probation}

Any member may bring a complaint to the Board and request that another member be placed on probation. The request must be reviewed at an open board meeting whose time and purpose are announced to the floor with at least 48 hours' advance notice. Any member may attend this meeting and speak on the topic of placing the member on probation. The floor member being considered may attend the meeting but not the vote; Board may decide to make the vote confidential.

A majority vote of Board is required to place a member on probation, at which time they must provide a timeline and requirements that must be met for the member to be let off probation. Additional meetings may be held to check on the member or re-evaluate the terms of probation.

Members who are on probation are expected to attend all meetings and activities, or else give a good reason for their absence. Members who are on probation but fail to comply with the terms will face expulsion.



	\subsection {Expulsion}

A member who violates probation is subject to expulsion. If the board determines that a member has violated probation, they must be given written, non-electronic notice that they have done so and that they must appear at the next board meeting to explain their actions. If they do not appear at the board meeting, having been given at least 48 hours' advance notice of its time and location, and cannot offer any explanation for their absence, they are expelled from CIF.

If the member offers an explanation of their actions to Board, then Board must hold a vote to decide if the explanation is acceptable. If four Board members vote that it is not, the member in question is expelled from CIF.

If a member remains on probation for an entire semester, they are brought to a hearing at a member meeting where they offer an explanation of their actions. The members will vote on the member's expulsion. If two-thirds of total membership or more vote in favor of expulsion, the member is expelled from CIF.

A member who has been expelled must be notified of their expulsion in person by the Chair. If the expulsion was voted on by Board, the Chair must ask the member if they wish to appeal the decision to the members. If they do not wish to appeal, their membership is immediately terminated. If they do wish to appeal, the appeal must be made at the next member meeting following all rules for overriding Board decisions. If the decision is not overridden, their membership is terminated at the end of the meeting.

If a resident member is expelled, they are expected to move off of floor in a timely manner. They must contact Residential Life about applying for a room change, with the assistance of the Chair, who is responsible for communicating and coordinating with the Residential Life Area Coordinator.

If a member is expelled before the halfway point of the semester, their dues for that semester will be refunded.



	\subsection {Withdrawal}

If a member decides that they no longer wish to be a member, they may withdraw from membership status. If a resident member decides to withdraw, they should arrange to move off floor in a timely manner.

If a member withdraws before the halfway point of the semester, their dues for that semester will be refunded.
